\documentclass[12pt]{beamer}
\usetheme{slidemath}

\usepackage[english]{babel}
\usepackage[utf8]{inputenc}
\usepackage{amsmath}
\usepackage{amsthm}
\usepackage{amsfonts}
\usepackage{amssymb}
\usepackage{calrsfs}
\usepackage{calligra}
\usepackage{calrsfs}
\usepackage[mathscr]{euscript}
%use graphics
\usepackage{graphicx}
\usepackage{tikz, tkz-base, tkz-fct}
\usepackage{wrapfig}

\usepackage{subcaption}
\usepackage{pgfplots}
\usepackage{adjustbox}


\title{Mathematical Slide}
\author{Author Name}
%\institution{Science Data}
\institute{Your Company or Institute}

\begin{document}
	\begin{frame}[plain]
		\maketitle
	\end{frame}

	
	\begin{frame}{Mathematical Slide}

		\begin{example}		
			conteúdo...
		\end{example}
		
		
		
		\begin{itemize}
			\item item 1
			\item item 2
		\end{itemize}
		
	\end{frame}

	\begin{frame}
		\begin{definition}{\textbf{Espaço Vetorial}}

		\end{definition}
	\end{frame}

	\begin{frame}
		\begin{theorem}
			conteúdo...
		\end{theorem}
	\end{frame}

\end{document}