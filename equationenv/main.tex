% -------------------------------------
% Templates
% Blog: https://wsricardo.blogspot.com
% https://github.com/mkvincis
% -------------------------------------
% Base for documents
\documentclass[a4,10pt]{article} % Type document (book, article, paper,)
\usepackage[portuguese]{babel} % Your language for document
%\usepackage[latin1]{inputenc}
\usepackage[utf8]{inputenc} % Complex characters
\usepackage[T1]{fontenc}
\usepackage{amsmath, amsfonts, amssymb}

\newtheorem{example}{Exemplo}

\begin{document}
	\begin{example}
		$	f(x) = \alpha x, \; \alpha \; \in \; \mathbb{R}$	
	\end{example}
	\begin{multline*}
	\begin{split}
	f^{-1}(x) = \frac{x}{\alpha} &\Rightarrow M_f(x_1, x_2) = f^{-1} \left( \frac{f(x_1) + f(x_2)}{2} \right)\\
	&\Rightarrow M_f(x_1, x_2) = f^{-1} \left( \frac{\alpha x_1 + \alpha x_2}{2} \right) \\
	&\boxed{M_f(x_1, x_2) = \frac{x_1 + x_2}{2}}, \; \text{Média Aritmética.}
	\end{split}
	\end{multline*}
	
\end{document}